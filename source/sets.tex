\chapter{Sets}
\section{Introduction}
Sets have important operations, such as the \vocab{cartesian product} $A \times B = \{ (a,b) : a \in A, b \in B \}$. 
\begin{proposition*}
$A \setminus B = A \cap B^c$. 
\end{proposition*}
\begin{proof}
	Notice that $x \in A\setminus B \Leftrightarrow x \in A, x \notin B \Leftrightarrow x \in A, x \in B^c \Leftrightarrow x \in A \cup B^c$. 
\end{proof}
Note that this makes sense intuitively, as the set minus operation is all elements $a \in A \notin B$, which would be $A \cap B^c$.
\begin{theorem}[DeMorgan's Laws] 
DeMorgan's Laws are the following:
\begin{enumerate}
\item $(A \cap B)^c = A^c \cup B^c$. 
\item $(A \cup B)^c = A^c \cap B^c$.
\end{enumerate} 
\end{theorem}
\begin{proof}
	Suppose $x \in (A\cap B)^c$. Then $x \notin A \cap B$. Since $x$ cannot be in both $A$ and $B$, suppose WLOG that $x\notin A$. Hence $x \in A^c \subseteq A^c \cup B^c$. Now suppose $x \in A^c \cup B^c$. THen $x \in A^c$ or $x \in B^c$, so suppose WLOG that $x \in A^c$. Hence $x \notin A$, so $x \notin A\cap B$. Thus $x \in (A\cap B)^c$. 

	Notice that $x \in (A\cup B)^c \Leftrightarrow x \notin (A\cup B) \Leftrightarrow x\notin A$ and $x \notin B \Leftrightarrow x \in A^c$ and $x \in B^c \Leftrightarrow x \in A^c \cap B^c$.
\end{proof}

\begin{definition}
A \vocab{function} $f: S \longrightarrow T$ is a map of each element $s \in S$ to \emph{at most} one element $t \in T$. $f$ can also be defined as a subset $F \subseteq S \times T$, such that each $s \in S$ occurs in at most one ordered pair in $F$. A function $f$ is \vocab{onto} or \vocab{surjective} if $\ran{f} = T$. $f$ is \vocab{one-to-one} or \vocab{injective} if $f(s_1) = f(s_2)$ implies $s_1 = s_2$. $f$ is a \vocab{bijection} from $S$ to $T$ if it is both injective and surjective. This also implies $f$ has an inverse. 
\end{definition}

\section{Bounded Sets}
\begin{definition}
A set $S$ is \vocab{bounded above} if there exists a number $b \in \mathbb{R}$ such that $s \leq b$ for all $s \in S$. Similarly, $S$ is \vocab{bounded below} if there is a number $\ell \in \mathbb{R}$ such that $s \geq \ell$ for all $s \in S$. 

A number $b_0 \in \mathbb{R}$ is the \vocab{supremum} of $S$ if it is the \emph{least} upper bound of $S$. It is denoted as $b_0 = \sup{S}$. Similarly, a number $\ell_0$ is the \vocab{infimum} of $S$ if it is the \emph{greatest} lower boundof $S$. It is denoted as $\ell_0 = \inf{S}$. Note that if $S$ contains its supremum, then we say that it is also the \vocab{maximum} of $S$. Similarly if $S$ contains its infimum, it is also the \vocab{minimum} of $S$. 
\end{definition}

\begin{theorem}[Supremum Property]
Every nonempty set of real numbers that is bounded above has a least upper bound. 
\end{theorem}
We take this as an axiom, but we can prove it later. 

\begin{proposition}
Assume $\mu \in \mathbb{R}$ is an upper bound for a set $S \subset \mathbb{R}$. Then $\mu = \sup{S} \Leftrightarrow$ for every $\varepsilon > 0$, there exists an element $s \in S$ such that $s \in [\mu-\varepsilon, \mu]$. 
\end{proposition}
\begin{proof}
	hi
\end{proof}<++>
Consequences of the supremum property are the following:
\begin{theorem}[Archimedian Property]
For any $b>0$, there are positive integers $n,m \in \mathbb{B}$ satisfying $\dfrac{1}{m} < b < n$. 
\end{theorem}
\begin{theorem}[Nested Interval Property]
For each $n \in \mathbb{N}$, assume we have a closed interval $I_n = [a_n, b_n]$. Assume these intervals are \emph{nested}: $I_{n+1} \subseteq I_n$. Then 
\[ \bigcap_{n=1}^\infty I_n \neq \emptyset.\]
\end{theorem}
\section{Cardinality, Countability, and Density}
We can use bijections to compare the cardinality of sets. 
\begin{definition}
Two sets have the same \vocab{cardinality} if there is a bijection $f: A \longrightarrow B$ $(A\sim B)$.
\end{definition}
Note that sets are either finite or infinite. But infinite sets do not necessarily have the same cardinality.
\begin{definition}
A set $A$ is \vocab{countable} if $\mathbb{N} \sim A$. If $A$ is an infinite set but $\mathbb{N} \nsim A$, then $A$ is \vocab{uncountable}. 
\end{definition}
We can establish some properties of countable sets. 
\begin{theorem}
If $S \subset T$ is infinite and $T$ is countable, then $S$ is also countable. 
\end{theorem}
\begin{theorem}
If $A_1, A_2, \cdots, A_n$ are each countable sets, then $\bigcup_{k=1}^n A_k$ is a countable set. If $A_n$ is a countable set for each $n \in \mathbb{N}$, then $\bigcup_{n=1}^\infty A_n$ is countable. 
\end{theorem}
\begin{theorem}
$\mathbb{Q}$ is countable. 
\end{theorem}
\begin{theorem}[Cantor]
The open interval $(0,1)$ is uncountable. 
\end{theorem}
\begin{proof}[Cantor's Diagonalization Argument]
Suppose there exists a bijection from $\mathbb{N} \longrightarrow (0,1)$. Then $1 \longrightarrow f(1) =  0.a_{11}a_{12}\cdots$, and so forth for all $n \in \mathbb{N}$. Now define $x \in (0,1) = 0.b_1b_2b_3\cdots$, where 
\[b_n =  
\begin{cases}
2 \text{ if } a_{nn} \neq 2 \\ 
3 \text{ if } a_{nn} = 2
\end{cases}
\]
Since no such $x = f(n)$ for any $n \in \mathbb{N}$, $(0,1)$ must be uncountable. 
\end{proof}
An immediate corollary of this is that $\mathbb{R}$ must be uncountable, as $(0,1) \subset \mathbb{R}$. Note that $\mathbb{R} = \mathbb{Q} \cup \mathbb{Q}^c$. Since $\mathbb{Q}$ is countable, $\mathbb{Q}^c$ must be uncountable.

\begin{theorem}[Density of $\mathbb{Q}$ in $\mathbb{R}$]
For every two real numbers $a$ and $b$ with $a<b$, there is a rational number $r$ such that $a<r<b$. 
\end{theorem}

\newpage

\section{Open and Closed Sets (Topology in $\mathbb{R}$)} 
\begin{definition}
Given $a \in \mathbb{R}$ and $\varepsilon > 0$, the \vocab{$\varepsilon$-neighborhood} of $a$ is the set $V_\varepsilon(a) = \{x \in \mathbb{R} : |x-a| < \varepsilon\}$. 
\end{definition}
\begin{definition}
A set $O \subset \mathbb{R}$ is \vocab{open} if for all points $p \in O$, there exists an $\varepsilon$-neighborhood $V_\varepsilon (p) \subset O$. 
\end{definition}
\begin{proposition}
If $A$ and $B$ are open, then $A \cup B$ is open. 
\end{proposition}
\begin{proposition}
If $A$ and $B$ are open, then $A \cap B$ is open. 
\end{proposition}
\begin{theorem}
If $A_1, A_2, \cdots$ are open, then $$ \bigcup_{\lambda \in S} A_\lambda $$ is open. The indexing set $S$ need not be countable. 
\end{theorem}
\begin{theorem}
If $A_1, \cdots, A_N$ are open, then 
$$ \bigcap_{n=1}^n A_n$$ is open.
\end{theorem}
Note that the above theorem does \emph{not} work when there are an infinite number of sets involved. 
\begin{definition}
A set $F\subset \mathbb{R}$ is \vocab{closed} if its complement $F^c$ is open. 
\end{definition}
\begin{theorem}
The union of a finite collection of closed sets is closed, and the intersection of an arbitraty collection of closed sets is closed.
\end{theorem}
The above theorem can be proved using DeMorgan's laws to shift the expressions to \vocab{Theorem 1.4.5} and \vocab{Theorem 1.4.6}'s statements.
\begin{definition}
A point $x \in \mathbb{R}$ is a \vocab{limit point} of $A$ if every $\varepsilon$-neighborhood of $x$ intersects the set $A$ at some point other than $x$. 
\end{definition}
\begin{theorem}
A set $F \subset \mathbb{R}$ is closed iff it contains its limit points. 
\end{theorem}
\begin{remark*}
Note that $\mathbb{R}$ and $\emptyset$ are both open and closed at the same time. These are the only two such sets in $\mathbb{R}$ that have this property. 
\end{remark*}
\section{The Cantor Set}
Consider the following sets: 
\begin{enumerate}
\item $C_1 = [0,1]$. 
\item $C_2 = C_1 \setminus (\frac{1}{3}, \frac{2}{3})$. 
\item $C_3 = C_2 \setminus [(\frac{1}{9}, \frac{2}{9}) \cup (\frac{7}{9}, \frac{8}{9})]$. 
\item So on and so forth for all $n \in \mathbb{N}$.
\end{enumerate}
\begin{definition}
The \vocab{Cantor set} is defined as $C = \displaystyle\bigcap_{n=1}^\infty C_n$. 
\end{definition}
Each closed interval in $C_n$ has length $(\frac{1}{3})^{n-1}$, so the total length of $C_n$ is $2^{n-1}(\frac{1}{3})^{n-1}$. Note that the length of $C$ is 0.
\begin{definition}
A set $A \subset \mathbb{R}$ is \vocab{perfect} if it is closed and contains no isolated points (all points are limit points). The Cantor set is one example of a perfect set. 
\end{definition}
The Cantor set is not empty, but contains an uncountable amount of points.  

